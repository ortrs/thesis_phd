\section{What is supercontinuum?}

Across literature, the concept of the supercontinuum resulted in the very first difficulty that proved a challenge to properly frame the topic of this thesis.

Alfano, in his seminal book "The Supercontinuum Laser Source: The Ultimate White Light" starts by defining supercontinuum as being "generated using ultrafast laser pulses propagating" and states a number of matter state and fiber types.

This proved to be a challenge at a first glance, as the general description of the phenomenon is quite far removed from the proposal of this very thesis:

* The laser source is not pulsed, but continuous wave
* The active material plays a significant role 
* The waveguide design is simplified

Reading onwards, we see more light (heh) shed onto the more detailed elements of the phenomenon: Supercontinuum gets its origin in self-phase modulation. Nonetheless, this effect is purely due to non-linear effects in materials, which in turn is related to the laser intensity. We claim that a continous-wave (CW) laser results sufficiently strong to excite white light, so the underlying effect may be different.
 
