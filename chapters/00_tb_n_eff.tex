\subsection{Effective Refractive Index}

In order to understand the phenomena occuring within the waveguides, it is important to first understand how waves are guided inside them.\par\medskip

Our understanding of waves is usually simplified to plane waves in vaccuum as this explanations provides the simplest frame of reference for wave propagation, as it considers the most ideal form of wave propagation: Homogeneous and isotropic media, meaning that there are no spatial dependencies of the wave propagation; lossless, meaning that there are no energy absorbers in the path of propagation of the wave; and linear, meaning that, for optical frequencies, the relationship between the polarization $\vec P$ of the wave and the electric field $\vec E$ of the interacting wave with the media.\par\medskip

From these assumptions, it is clearer to understand where the drawbacks of such an idealized model relies: For real fabricated waveguides, their surface will generally not be perfectly smooth, leading to surface roughness which was investigated by Vlasov, et. Al. ~\cite{Vlasov:04} for silicon-on-insulator (SOI) platforms. Additionally, the propagation modes in non-ideal media cannot necessarily be described by pure TE- or TM-modes, leading to the so-called quasi-TE or -TM modes, as the mode confinement is not perfectly matched to the waveguide. Finally, and most important to this thesis, the waveguide will guide a quasi-TE or -TM mode that mostly travels through the surface of the waveguide, as the intended interaction volume will be around the surface of the slab or the slot, as does the reference work from Korn, et. Al. ~\cite{Korn:16}, due to a similar effect to what is typically observed in near-field enhancement in noble metals (to the best of my knowledge, 
